\documentclass{article}
\usepackage{natbib}
\usepackage{graphicx}
% ------------------------------------------------------------------------
% Packages
% ------------------------------------------------------------------------
\usepackage{amsmath}

% ------------------------------------------------------------------------
% Macros
% ------------------------------------------------------------------------
%~~~~~~~~~~~~~~~
% List shorthand
%~~~~~~~~~~~~~~~
\newcommand{\BIT}{\begin{itemize}}
\newcommand{\EIT}{\end{itemize}}
\newcommand{\BNUM}{\begin{enumerate}}
\newcommand{\ENUM}{\end{enumerate}}
%~~~~~~~~~~~~~~~
% Text with quads around it
%~~~~~~~~~~~~~~~
\newcommand{\qtext}[1]{\quad\text{#1}\quad}
%~~~~~~~~~~~~~~~
% Shorthand for math formatting
%~~~~~~~~~~~~~~~
\newcommand\mbb[1]{\mathbb{#1}}
\newcommand\mbf[1]{\mathbf{#1}}
\def\mc#1{\mathcal{#1}}
\def\mrm#1{\mathrm{#1}}
%~~~~~~~~~~~~~~~
% Common sets
%~~~~~~~~~~~~~~~
\def\reals{\mathbb{R}} % Real number symbol
\def\integers{\mathbb{Z}} % Integer symbol
\def\rationals{\mathbb{Q}} % Rational numbers
\def\naturals{\mathbb{N}} % Natural numbers
\def\complex{\mathbb{C}} % Complex numbers
\def\simplex{\mathcal{S}} % Simplex
%~~~~~~~~~~~~~~~
% Common functions
%~~~~~~~~~~~~~~~
\renewcommand{\exp}[1]{\operatorname{exp}\left(#1\right)} % Exponential
\def\indic#1{\mbb{I}\left({#1}\right)} % Indicator function
\providecommand{\argmax}{\mathop\mathrm{arg max}} % Defining math symbols
\providecommand{\argmin}{\mathop\mathrm{arg min}}
\providecommand{\arccos}{\mathop\mathrm{arccos}}
\providecommand{\asinh}{\mathop\mathrm{asinh}}
\providecommand{\dom}{\mathop\mathrm{dom}} % Domain
\providecommand{\range}{\mathop\mathrm{range}} % Range
\providecommand{\diag}{\mathop\mathrm{diag}}
\providecommand{\tr}{\mathop\mathrm{tr}}
\providecommand{\abs}{\mathop\mathrm{abs}}
\providecommand{\card}{\mathop\mathrm{card}}
\providecommand{\sign}{\mathop\mathrm{sign}}
\def\rank#1{\mathrm{rank}({#1})}
\def\supp#1{\mathrm{supp}({#1})}
%~~~~~~~~~~~~~~~
% Common probability symbols
%~~~~~~~~~~~~~~~
\def\E{\mathbb{E}} % Expectation symbol
\def\Earg#1{\E\left[{#1}\right]}
\def\Esubarg#1#2{\E_{#1}\left[{#2}\right]}
\def\P{\mathbb{P}} % Probability symbol
\def\Parg#1{\P\left({#1}\right)}
\def\Psubarg#1#2{\P_{#1}\left[{#2}\right]}
\def\Cov{\mrm{Cov}} % Covariance symbol
\def\Covarg#1{\Cov\left[{#1}\right]}
\def\Covsubarg#1#2{\Cov_{#1}\left[{#2}\right]}
\def\Var{\mrm{Var}}
\def\Vararg#1{\Var\left(#1\right)}
\def\Varsubarg#1#2{\Var_{#1}\left(#2\right)}
\newcommand{\family}{\mathcal{P}} % probability family
\newcommand{\eps}{\epsilon}
\def\absarg#1{\left|#1\right|}
\def\msarg#1{\left(#1\right)^{2}}
\def\logarg#1{\log\left(#1\right)}
%~~~~~~~~~~~~~~~
% Distributions
%~~~~~~~~~~~~~~~
\def\Gsn{\mathcal{N}}
\def\Ber{\textnormal{Ber}}
\def\Bin{\textnormal{Bin}}
\def\Unif{\textnormal{Unif}}
\def\Mult{\textnormal{Mult}}
\def\Cat{\textnormal{Cat}}
\def\Gam{\textnormal{Gam}}
\def\InvGam{\textnormal{InvGam}}
\def\NegMult{\textnormal{NegMult}}
\def\Dir{\textnormal{Dir}}
\def\Lap{\textnormal{Laplace}}
\def\Bet{\textnormal{Beta}}
\def\Poi{\textnormal{Poi}}
\def\HypGeo{\textnormal{HypGeo}}
\def\GEM{\textnormal{GEM}}
\def\BP{\textnormal{BP}}
\def\DP{\textnormal{DP}}
\def\BeP{\textnormal{BeP}}
%~~~~~~~~~~~~~~~
% Theorem-like environments
%~~~~~~~~~~~~~~~

%-----------------------
% Probability sets
%-----------------------
\newcommand{\X}{\mathcal{X}}
\newcommand{\Y}{\mathcal{Y}}
\newcommand{\D}{\mathcal{D}}
\newcommand{\Scal}{\mathcal{S}}
%-----------------------
% vector notation
%-----------------------
\newcommand{\bx}{\mathbf{x}}
\newcommand{\by}{\mathbf{y}}
\newcommand{\bt}{\mathbf{t}}
\newcommand{\xbar}{\overline{x}}
\newcommand{\Xbar}{\overline{X}}
\newcommand{\tolaw}{\xrightarrow{\mathcal{L}}}
\newcommand{\toprob}{\xrightarrow{\mathbb{P}}}
\newcommand{\laweq}{\overset{\mathcal{L}}{=}}
\newcommand{\F}{\mathcal{F}}


\title{Notes}
\author{Kris Sankaran}
\maketitle

\begin{document}

\subsection{Counting binary trees}

\begin{itemize}
  \item Want $\beta_{n}:=\#\left\{\text{binary trees on } n \text{ vertices}\right\}$.
  \item Since subtrees are still binary trees, determine recurrence and use generating functions.
  \item By dividing into cases for the number of vertices in each subtree, derive the recurrence,
    \begin{align}
      \beta_{0} = 1 \\
      \beta_{n} = \sum_{i = 0}^{n - 1} \beta_{i}\beta_{n - i - 1}
    \end{align}
  \item Relating this to the earlier problem of counting ways to triangulate $n$-gons, we get
    \begin{align}
      \beta_{n} = \frac{1}{n}{2n \choose n},
    \end{align}
    which is coincidentally the $n^{th}$ catalan number.
\end{itemize}
    
\subsection{Coloring and the chromatic polynomial}


\begin{itemize}
  \item What's the fewest number of colors so that we can color each vertex in a
    graph so that neighbors always get different colors? This number
    $\chi\left(G\right)$ is the chromatic number of $G$.
  \item If the graph is 2 colorable, it's bipartite.
  \item The chromatic number for a complete graph is the total number of vertices.
  \item Arguments for lower bounds are usually attempts to label vertices one by
    one until you arrive at an edge with two colors that are necessarily the
    same.
  \item The chromatic polynomial is defined as $p\left(G, k\right) =
    \#\{\text{proper }k\text{-colorings of }G\}$.
  \item Example: To $k$-color a graph, you can choose any of $k$ colors for
    the first node, then for each subsequent one you can choose any of $k -
    1$ colors (any but the previous one). So the associated polynomial is
    $k\left(k - 1\right)^{n - 1}$.
  \item Since counting these colorings is a type of ``pattern-avoidance''
    problem, it is often useful to use inclusion-exclusion, where you're
    trying to avoid any edge having endpoints the same color.
  \item This can require $2^{e\left(G\right)}$ evaluations of $N_{\geq}$
    though, so it's useful to study recurrences.
  \item Define a contraction operator $\cdot$ that merges two vertices connected
    by an edge. Then you can show
    \begin{align}
      p\left(G, k\right) = p\left(G - e, k\right) - p\left(G \cdot e, k\right),
    \end{align}
    by double counting $p\left(G - e, k\right)$, conditioning on whether the
    endpoints of $e$ are the same color or not.
  \item These polynomials have some other nice properties: the leading
    coefficient is always 1, the constant's coefficient is 0, the coefficient of
    $k^{n - 1}$ is $-e\left(G\right)$ (minus the number of edges of $G$),
    the signs alternate, and the sum of coefficients is 0. Last property has a
    nice proof: just note that $p\left(G, 1\right) = 0$ since if you have at
    least one edge you can't be 1-colorable.
\end{itemize}

\subsection{Ramsey theory}

\begin{itemize}
  \item General flavor: In a large structure that is constructed arbitrarily,
    there is some concrete nonrandom structure.
  \item Typical problem: What is the smallest $n$ so that any red-blue coloring
    of edges in $K_{n}$, the complete graph on $n$ vertices, contains either a
    red $K_{a}$ or a blue $K_{b}$. This number is called $R\left(a, b\right)$.
    People sometimes write $n \arrow a, b$ if $R\left(a, b\right) = n$.
  \item There's a clever proof for $R\left(3, 3\right)$, it starts by
    considering three edges that must be one color and then analyzing what
    happens on the remaining edges.
  \item One approach to constructing upper bounds is to note $R\left(a, b\right)
    \leq R\left(a - 1, b\right) + R\left(a, b - 1\right)$. The proof considers
    edges incident with the first node, and dividing into the cases there are
    $R\left(a - 1, b\right)$ red edges or $R\left(a, b - 1\right)$ blue edges
    incident to it.
  \item You can generalize ramsey numbers to have more colors, or to subgraphs
    other than complete graphs.
\end{itemize}

\section{Designs and codes}

\subsection{Construction methods for designs}

\begin{itemize}
  \item A combinatorial design on a set $V$ is a collection of subsets $\B$. The
    elements of $V$ are called varieties.
    \item A $\left(b, v, r, k, \lambda\right)$ design has $b$ blocks, $v$
      variaties, $r$ blocks per each variety, $k$ varieties in each block, and
      $lambda$ blocks including any given pair of varieties. If $k = v$ the
      design is complete.
\end{itemize}

\end{document}
