\documentclass{article}
\usepackage{natbib}
\usepackage{graphicx}
\input{preamble.tex}

\title{Notes on ``Data Analysis for Network Cybersecurity''}
\author{Kris Sankaran}

\begin{document}
\section{Inference for Graphs and Networks}
\label{sec:chapter1}

$A$ is the adjacency matrix. $c\left(i\right)$ are covariates associated with
the $i^{th}$ node. This is depressing, people are getting hacked and all we can
write about is the karate network??

Definition of Erdos-Renyi: $A_{ij} \sim \Ber\left(p\right)$ for $i < j$ and then
symmetrize. No self-loops. This is often treated as a kind of null. A natural
alternative is that there is block structure, e.g.,
$A_{ij} \sim \Ber\left(p_{c\left(i\right)}p_{c\left(j\right)}\right)$ defines
the standard stochastic blockmodel.

Estimation fo the underlying $p_{c\left(i\right)c\left(j\right)}$ is
combinatorially hard when the blocks are unobserved, but you can get pretty good
approximate answers using the eigenvalues of the laplacian $L = D - A$ where $D$
is diagonal with the degrees of the nodes. The point is that if there are
actually $K$ clusters, then $K$ of the eigenvalues will be zero.

They use a $\chi^{2}$ test to see whether the inter and intra subgroup link
probabilities are equal to each other. You can also do a generalized likelihood
ratio of null erdos-renyi vs. stochastic blockmodel, usign the fact that MLE can
be computed from the spectral clustering approximation.

An issue is that the statistics so far have only depended on degree
distributions. This is a very coarse view of a network. An alternative is to
sample from graphs under the null with an apriori fixed degree distribution, and
then compute whatever other statistics on this.

One issue swept under the rug is that many edges or nodes might not actually be
observed. There doesn't seem to be a good way of dealing with this.

There are no applications to cybersecurity here, pretty disappointing.

\end{document}
