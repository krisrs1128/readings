\documentclass{article}
\usepackage{natbib}
\usepackage{graphicx}
\input{preamble.tex}

\title{Notes on Sampling NBP with Augmentation}
\author{Kris Sankaran}

\begin{document}

This is a brief summary of \citep{zhou2012augment}, just for future reference.
The overall goal of the paper is to leverage a relatively less well-known
representation of the negative binomial distribution in order to facilitate
sampling of a NBP with a hierarchical GaP prior on the overdispersion parameter.
This model has an interpretation as the unnormalized, completely random measure
analog of the hierarchical dirichlet process \citep{teh2005sharing}. The main
advantage of the NBP here is that, when properly augmented, it's possible to do
exact gibbs sampling, which makes implementation more straightforwards than for
the HDP. One point not emphasized so much in the paper, but which is how their
sampler is actually implemented, is that their exact sampler is for a
hierarchical dirichlet distribution with large $K$, not actually an HDP. Since
the hierarchical dirichlet distribution is actually pretty useful in practice
\citep{wallach2006topic}, I have no problem with this, but just be aware that
some of the advertising in the paper is a little misleading. Anyways, onto the
actual summary.

The first main point is that an elementary property of marked poisson processes
allows a connection between count and mixture modeling. Informally, count
modeling is reflected by the counts of arrivals in a poisson process. If we mark
each arrival with color $k$ with probability $p_{k}$ and then condition on the
total number of arrivals, the resulting distribution is multinomial with
probabilities $p_{k}$ -- i.e., the conditioned process can be interpreted as a
mixture model. Using the notation of the paper, if $X_{j}\left(\cdot\right)$ are
$J$ different $\PoiP\left(G\right)$ ($G$ is a potentially inhomegenous rate
function), then for any partition $A_{1}, \dots, A_{q}$, we have for each $j$

\begin{align}
  \left(X_{j}\left(A_{1}\right), \dots, X_{j}\left(A_{q}\right)\right) \vert X_{j}\left(\Omega\right) \sim \Mult\left(\X_{j}\left(\Omega\right), \tilde{G}\left(A_{1}\right), \dots, \tilde{G}\left(A_{q}\right)\right),
\end{align}
where $\tilde{G}$ is the normalized rate measure, $tilde{G} =
\frac{G}{G\left(\Omega\right)}$.

\bibliographystyle{plainnat}
\bibliography{refs.bib}

\end{document}
