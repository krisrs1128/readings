\documentclass{article}
\usepackage{natbib}
\usepackage{amssymb, amsmath}
\usepackage{graphicx}
% ------------------------------------------------------------------------
% Packages
% ------------------------------------------------------------------------
\usepackage{amsmath}

% ------------------------------------------------------------------------
% Macros
% ------------------------------------------------------------------------
%~~~~~~~~~~~~~~~
% List shorthand
%~~~~~~~~~~~~~~~
\newcommand{\BIT}{\begin{itemize}}
\newcommand{\EIT}{\end{itemize}}
\newcommand{\BNUM}{\begin{enumerate}}
\newcommand{\ENUM}{\end{enumerate}}
%~~~~~~~~~~~~~~~
% Text with quads around it
%~~~~~~~~~~~~~~~
\newcommand{\qtext}[1]{\quad\text{#1}\quad}
%~~~~~~~~~~~~~~~
% Shorthand for math formatting
%~~~~~~~~~~~~~~~
\newcommand\mbb[1]{\mathbb{#1}}
\newcommand\mbf[1]{\mathbf{#1}}
\def\mc#1{\mathcal{#1}}
\def\mrm#1{\mathrm{#1}}
%~~~~~~~~~~~~~~~
% Common sets
%~~~~~~~~~~~~~~~
\def\reals{\mathbb{R}} % Real number symbol
\def\integers{\mathbb{Z}} % Integer symbol
\def\rationals{\mathbb{Q}} % Rational numbers
\def\naturals{\mathbb{N}} % Natural numbers
\def\complex{\mathbb{C}} % Complex numbers
\def\simplex{\mathcal{S}} % Simplex
%~~~~~~~~~~~~~~~
% Common functions
%~~~~~~~~~~~~~~~
\renewcommand{\exp}[1]{\operatorname{exp}\left(#1\right)} % Exponential
\def\indic#1{\mbb{I}\left({#1}\right)} % Indicator function
\providecommand{\argmax}{\mathop\mathrm{arg max}} % Defining math symbols
\providecommand{\argmin}{\mathop\mathrm{arg min}}
\providecommand{\arccos}{\mathop\mathrm{arccos}}
\providecommand{\asinh}{\mathop\mathrm{asinh}}
\providecommand{\dom}{\mathop\mathrm{dom}} % Domain
\providecommand{\range}{\mathop\mathrm{range}} % Range
\providecommand{\diag}{\mathop\mathrm{diag}}
\providecommand{\tr}{\mathop\mathrm{tr}}
\providecommand{\abs}{\mathop\mathrm{abs}}
\providecommand{\card}{\mathop\mathrm{card}}
\providecommand{\sign}{\mathop\mathrm{sign}}
\def\rank#1{\mathrm{rank}({#1})}
\def\supp#1{\mathrm{supp}({#1})}
%~~~~~~~~~~~~~~~
% Common probability symbols
%~~~~~~~~~~~~~~~
\def\E{\mathbb{E}} % Expectation symbol
\def\Earg#1{\E\left[{#1}\right]}
\def\Esubarg#1#2{\E_{#1}\left[{#2}\right]}
\def\P{\mathbb{P}} % Probability symbol
\def\Parg#1{\P\left({#1}\right)}
\def\Psubarg#1#2{\P_{#1}\left[{#2}\right]}
\def\Cov{\mrm{Cov}} % Covariance symbol
\def\Covarg#1{\Cov\left[{#1}\right]}
\def\Covsubarg#1#2{\Cov_{#1}\left[{#2}\right]}
\def\Var{\mrm{Var}}
\def\Vararg#1{\Var\left(#1\right)}
\def\Varsubarg#1#2{\Var_{#1}\left(#2\right)}
\newcommand{\family}{\mathcal{P}} % probability family
\newcommand{\eps}{\epsilon}
\def\absarg#1{\left|#1\right|}
\def\msarg#1{\left(#1\right)^{2}}
\def\logarg#1{\log\left(#1\right)}
%~~~~~~~~~~~~~~~
% Distributions
%~~~~~~~~~~~~~~~
\def\Gsn{\mathcal{N}}
\def\Ber{\textnormal{Ber}}
\def\Bin{\textnormal{Bin}}
\def\Unif{\textnormal{Unif}}
\def\Mult{\textnormal{Mult}}
\def\Cat{\textnormal{Cat}}
\def\Gam{\textnormal{Gam}}
\def\InvGam{\textnormal{InvGam}}
\def\NegMult{\textnormal{NegMult}}
\def\Dir{\textnormal{Dir}}
\def\Lap{\textnormal{Laplace}}
\def\Bet{\textnormal{Beta}}
\def\Poi{\textnormal{Poi}}
\def\HypGeo{\textnormal{HypGeo}}
\def\GEM{\textnormal{GEM}}
\def\BP{\textnormal{BP}}
\def\DP{\textnormal{DP}}
\def\BeP{\textnormal{BeP}}
%~~~~~~~~~~~~~~~
% Theorem-like environments
%~~~~~~~~~~~~~~~

%-----------------------
% Probability sets
%-----------------------
\newcommand{\X}{\mathcal{X}}
\newcommand{\Y}{\mathcal{Y}}
\newcommand{\D}{\mathcal{D}}
\newcommand{\Scal}{\mathcal{S}}
%-----------------------
% vector notation
%-----------------------
\newcommand{\bx}{\mathbf{x}}
\newcommand{\by}{\mathbf{y}}
\newcommand{\bt}{\mathbf{t}}
\newcommand{\xbar}{\overline{x}}
\newcommand{\Xbar}{\overline{X}}
\newcommand{\tolaw}{\xrightarrow{\mathcal{L}}}
\newcommand{\toprob}{\xrightarrow{\mathbb{P}}}
\newcommand{\laweq}{\overset{\mathcal{L}}{=}}
\newcommand{\F}{\mathcal{F}}


\title{Notes on Sampling NBP with Augmentation}
\author{Kris Sankaran}

\begin{document}

This is a brief summary of \citep{zhou2012augment}, just for future reference.
The overall goal of the paper is to leverage a relatively less well-known
representation of the negative binomial distribution in order to facilitate
sampling of a NBP with a hierarchical GaP prior on the overdispersion parameter.
This model has an interpretation as the unnormalized, completely random measure
analog of the hierarchical dirichlet process \citep{teh2005sharing}. The main
advantage of the NBP here is that, when properly augmented, it's possible to do
exact gibbs sampling, which makes implementation more straightforwards than for
the HDP. One point not emphasized so much in the paper, but which is how their
sampler is actually implemented, is that their exact sampler is for a
hierarchical dirichlet distribution with large $K$, not actually an HDP. Since
the hierarchical dirichlet distribution is actually pretty useful in practice
\citep{wallach2006topic}, I have no problem with this, but just be aware that
some of the advertising in the paper is a little misleading. Anyways, onto the
actual summary.

\section{Observation \#1 (Philosophical)}

The first main point is that an elementary property of marked poisson processes
allows a connection between count and mixture modeling. Informally, count
modeling is reflected by the counts of arrivals in a poisson process. If we mark
each arrival with color $k$ with probability $p_{k}$ and then condition on the
total number of arrivals, the resulting distribution is multinomial with
probabilities $p_{k}$ -- i.e., the conditioned process can be interpreted as a
mixture model. Using the notation of the paper, if $X_{j}\left(\cdot\right)$ are
$J$ different $\PoiP\left(G\right)$ ($G$ is a potentially inhomegenous rate
function), then for any partition $A_{1}, \dots, A_{q}$, we have for each $j$

\begin{align}
  \left(X_{j}\left(A_{1}\right), \dots, X_{j}\left(A_{q}\right)\right) \vert X_{j}\left(\Omega\right) \sim \Mult\left(\X_{j}\left(\Omega\right), \tilde{G}\left(A_{1}\right), \dots, \tilde{G}\left(A_{q}\right)\right),
\end{align}
where $\tilde{G}$ is the normalized rate measure, $\tilde{G} =
\frac{G}{G\left(\Omega\right)}$.

\section{Observation \#2 (Mathematical)}

The essential research idea of this paper depends on an
elementary-but-I-had-never-seen-before probability fact. Let $m \sim NB\left(r,
p\right)$, a negative binomial with overdispersion\footnote{Confusingly, the
  associated overdispersion is $r^{-1}$.} parameter $r$ and probability $p$. Then, there is as representation,
\begin{align}
  m \vert l &\sim \sum^{l} \Log\left(p\right) \\
  l &\sim \Poi\left(-r \log\left(1 - p\right)\right),
\end{align}
where the sum in the first line is shorthand for the sum of $l$ i.i.d.
Logarithmically distributed variables. There is actually a connection between
this representation and the chinese restaurant process -- $l$ can be thought of
as the number of tables at which a total number of $m$ customers are seated, but
this is not the focus of this paper, see \citep{zhou2015negative} instead.

Now, recall the more standard fact that $m$ can also be represented as
\begin{align}
  m \vert \lambda &\sim \Poi\left(\lambda\right) \\
  \lambda &\sim \Gamma\left(r, \frac{p}{1 - p}\right).
\end{align}

Combining these facts, we can get a nice representation of a negative binomial
with a gamma prior on the overdispersion parameter, more precisely, for $m \sim
NB\left(r, p\right)$ where $r \sim \Gamma\left(r_{1}, \frac{1}{c_{1}}\right)$.
The first step is to write
\begin{align}
  m \vert l &\sim \sum^{l} \Log\left(p\right) \\
  l \vert r &\sim \Poi\left(-r\log\left(1 - p\right)\right) \\
  r &\sim \Gamma\left(r_{1}, \frac{1}{c_{1}}\right).
\end{align}
Marginalizing out $r$, we see that $l$ is also negative binomial. So, we can use
the logarithmic representation again, which gives
\begin{align}
  m \vert l &\sim \sum^{l} \Log\left(p\right) \\
  l \vert l^{\prime} &\sim \sum^{l^{\prime}} \Log\left(p^{\prime}\right) \\
  l^{\prime} &\sim \Poi\left(-r_{1} \log\left(1 - p^{\prime}\right)\right),
\end{align}
where $p^{\prime} = \frac{-\log\left(1 - p\right)}{c - \log\left(1 - p\right)}$.

\section{How these are useful}

These calculations are not complicated, but they are definitely not obvious, and
their implications are subsantial. First of all, by making the same arguments as
above across partitions of the sample space, we can get a negative binomial
process analog of the representation in the previous section,
\begin{align}
  X_{j} \vert G &\sim \NBP\left(G, p_{j}\right) \\
  G &\sim \GaP\left(c, G_{0}\right)
\end{align}
is equivalent to
\begin{align}
  X_{j} \vert \Lambda_{j} &\sim \PoiP\left(\Lambda_{j}\right) \\
  \Lambda_{j} \vert G &\sim \GaP\left(\frac{1 - p_{j}}{p_{j}}, G\right) \\
  G &\sim \GaP\left(c, G_{0}\right),
\end{align}
which in turn is equivalent to
\begin{align}
  X_{j} \vert L_{j} &\sim \sum^{L_{j}} \Log\left(p_{j}\right) \\
  L_{j} \vert L^{\prime} &\sim \sum^{L^{\prime}} \Log\left(p^{\prime}\right) \\
  L^{\prime} &\sim \PoiP\left(-G_{0}\log\left(1 - p^{\prime}\right)\right),
\end{align}
where $p^{\prime}$ is the same transformation of $p$ discussed before. Think of
the $L_{j}$ as per-corpus dirichlet process mixture weights, while $L^{\prime}$
defines the shared-across-corpuses dirichlet base measure in the HDP (of course,
these are not actually dirichlet processes, though if you consider normalizing
the $G$'s you can see the connection).

\subsection{LDA analog}

The point of this representation of the NBP prior is that allows gibbs sampling
in many natural nonparameteric mixture modeling scenarios. Consider the case
that the base measure $G_{0}$ is discrete with $K$ atoms $\omega_{1}, \dots,
\omega_{K}$. Then, the NBP models listed above reduce to hierarchical versions
of common discrete distributions (this is the reason for the cavet mentioned in
the intro). For example, $G_{0}$ can now be written as $\sum_{k = 1}^{K} r_{k}
\delta_{\omega_{k}}$ for $r_{k}$ drawm from a $\Gamma\left(\frac{\gamma_{0}}{K},
\frac{1}{c}\right)$.

In this setting, it is natural to consider an analog of LDA. Let $x_{ji} \sim
F\left(\omega_{z_{ji}}\right)$ be the term associated with the $i^{th}$ word in
the $j^{th}$ document. Using an NBP prior to assign topics $z_{ji}$, and using a
Dirichlet distribution to define the positions of atoms in the base measure, we
have
\begin{align}
  x_{ji} \vert z_{ji}, \left(\omega_{k}\right) &\sim F\left(\omega_{z{ji}}\right) \\
  \omega_{k} &\sim \Dir\left(\eta 1_{V}\right) \\
  N_{j} \vert \left(\lambda_{jk}\right)&= \sum_{k = 1}^{K} \Poi\left(\lambda_{jk}\right) \\
  \lambda_{jk} \vert r_{j}, p_{j} &\sim \Gamma\left(r_{j}, \frac{p_{j}}{1 - p_{j}}\right) \\
  r_{j} &\sim \Gamma\left(\gamma_{0}, \frac{1}{c}\right)
\end{align}
and less crucially, we can place conjugate priors on some of hte remaining
variables,
\begin{align}
  p_{j} &\sim \Bet\left(a_{0}, p_{0}\right) \\
  \gamma_{0} &\sim \Gamma\left(e_{0}, \frac{1}{f_{0}}\right).
\end{align}
We can interpret the relative size of $\lambda_{jk}$'s as the ``amount'' of
topic $k$ in document $j$. Note that we have put a natural conjugate prior on
the $p_{j}$, but there is otherwise nothing special about this parameter (this
is a major difference between this paper and \cite{broderick2015combinatorial}).

Augmenting this description with the logarithmically distributed variables
described before, the complete conditionals are then all available in closed
form,
\begin{align}
  p_{j} \vert \dots &\sim \Bet\left(a_{0} + N_{j}, b_{0} + \sum r_{k}\right) \\
  l_{jk} \vert \dots &\sim \CRT\left(n_{jk}, r_{k}\right) \\
  l^{\prime}_{k} \vert \dots &\sim \CRT\left(\sum l_{jk}, \frac{\gamma_{0}}{K}\right) \\
  \gamma_{0} \vert \dots &\sim \Gamma\left(e_{0} + \sum l^{\prime}_{k}, \frac{1}{f_{0} - \log\left(1 - p^{\prime}\right)}\right) \\
  r_{k} \vert \dots &\sim \Gamma\left(\frac{\gamma_{0}}{K} + \sum l_{jk}, \frac{1}{c - \sum \log\left(1 - p_{j}\right)}\right) \\
  \lambda_{jk} \vert \dots &\sim \Gamma\left(r_{k} + n_{jk}, p_{j}\right).
\end{align}


Note that we could alternatively work in terms of the $N_{j}$'s, so we only
really need word counts. This just collapses the state-space for the sampler (we
no longer need the $z_{ji}$'s).

They note a few extensions of this model,
\begin{itemize}
\item It's possible to put a beta process prior on the $p_{j}$'s, to encourage sharing among them.
  \item Instead of $r_{j}$'s you can use $RZ_{j}$ where the $Z_{j}$'s are
    bernoulli, to induce zero inflation.
\end{itemize}

\subsection{Zero-Inflation}

Since zero-inflation is of direct interest to our lab, we write the model in
more detail. Inference is described in the appendix of the paper, and is very
similar to the description written above. The main difference is that there is a
$b_{jk}$ binary variable that specifies whether a particular topic ever appears
within a given document.

\begin{align}
  n_{jk} \vert \lambda_{jk} &\sim \Poi\left(\lambda_{jk}\right) \\
  \lambda_{jk} \vert b_{jk}, r_{j}, p_{j} &\sim \Gamma\left(r_{k}b_{jk}, \frac{p_{j}}{1 - p_{j}}\right) \\
  r_{k} &\sim \Gamma\left(\gamma_{0}, \frac{1}{c}\right) \\
  b_{jk} &\sim \Ber\left(\pi_{k}\right) \\
  \pi_{k} &\sim \Bet\left(\frac{c}{K}, c\left(1 - \frac{1}{K}\right)\right)
\end{align}

\bibliographystyle{plainnat}
\bibliography{refs.bib}
\end{document}
