\documentclass{article}
\usepackage{natbib}
\usepackage{graphicx}
\input{preamble.tex}

\title{Experiments with Nonparametric ANOVA}
\author{Kris Sankaran}

\begin{document}

Can we try to understand the method described in \citep{anderson2001new} from a
careful statistical point of view? In what settings is it appropriate? Is it
possible to characterize \textit{why} hypotheses get rejected, and provide a
sense of effect sizes, rather than focusing only on overall significance? Though
the method is based on ANOVA -- one of the most well-understood techniques in
statistics -- there remains something of an air of mystery surrounding ADONIS,
especially with the sort of distances and data that appear frequently in
microbiome studies. The goal of this note is to brainstorm and summarize some
simulation and mathetical experiments towards understanding the behavior of this
method.

\section{Simulation ideas}

\subsection{Under the null}

By definition, under the null, the $p$-values created by ADONIS are uniform.
Nonetheless, it is still worth studying the permutation distribution of the
distance-based $F$-statistic under different data-generation scenarios. For
example, it would be useful to know whether there are any types of concentration
phenomena that arise under different $\frac{p}{n}$ regimes or how highly
non-normal, zero-inflated, or outlier-contaminated data affect the null.

Here is a list of possible experimental directions.

\subsection{Under alternatives}

\bibliographystyle{plainnat}
\bibliography{refs.bib}

\end{document}
