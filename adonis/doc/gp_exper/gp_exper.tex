\documentclass{article}
\usepackage{natbib}
\usepackage{graphicx}
\input{preamble.tex}
\linespread{1.5}

\title{Invalidity of ADONIS in the Pesence of Between-Site Correlation}
\author{Kris Sankaran}

\begin{document}
\maketitle

On occassion, the $p$-values obtained by the permutation procedure described in
\citep{anderson2001new} (implemented as the \texttt{adonis} function in
\cite{oksanen2007vegan}) have seemed suspiciously significant, sometimes out of
line with several other analysis on the same data. We have had various
hypotheses about why the permutation procedure might not be appropriate in
microbiome data -- bacteria and sites tend to be correlated across phyla and
time, for example. In this note, we try to formalize this intuition and quantify
the robustness (or lack thereof) of the adonis permutation procedure. We believe
that a more careful understanding of this method is a relevant area of
statistical research, in light of the popularity of this method in standard
microbiome analysis practice (see e.g. \cite{fukuyama2012comparisons} and
references therein).

Our analysis is directly inspired by \citep{guillot2013dismantling}, which also
studied validity of the Mantel test, another type of distanced-based,
nonparametric hypothesis test. The main idea in both their work and ours is that
measurements across multiple sites will often express some sort of between-site
correlation structure, and that it is important to maintain this structure under
the permutation null. More precisely, the null hypothesis should only be that
the two processes are independent of one another, not that the two processes are
independent and each have no between-site correlation. But the latter structure
is what is simulated in ADONIS' permutation null, and if the processes are
actually independent but with within-process site correlation, ADONIS will still
report that there is an association between them, when there are in fact none.

We will describe some simple linear-model based alternatives to ADONIS that
provide valid $p$-values. The drawback of these approaches is that they cannot
be applied when only distances between sites is available -- a setting which is
meaningful in microbiome studies, where working entirely with UniFrac distances
\citep{lozupone2005unifrac} is common, for example.

There are two main limitations in this note,

\begin{itemize}
\item As in any simulation study, the conclusions are only as useful as the
  simulation mechanism is representative of the data encountered in practice.
  Ideally we could characterize the plausibility of our simulation mechanism,
  with respect to real data.
\item We have no alternative to ADONIS that can be applied when only distances
  are available. We could imagine more sophisticated approaches, based on
  model-free knockoffs, for example.
\end{itemize}

Hopefully these deficiencies will be resolved as this work matures.

\section{Setup}
\label{sec:setup}

In this section, we describe the our simulation design. This has two components:
how do we generate the null data, and how do we test for associations. We have
several approaches for each.

\subsection{Data generating mechanism}
\label{subsec:data_generating_mechanism}

As mentioned in the introduction, the goal of our simulation study is to
formally encode the qualitative observation that ADONIS may incorrectly report
an association between independent measurements when the measurements exhibit
between-site correlation. Suppose that we have populated a matrix
$X \in \reals^{n \times p}$ of abundances for $p$ species across $n$ sites. A vector
$y \in \{0, 1\}^{n}$ describes a binary characteristic at each site.

This 

- What models do we consider simulating under?
- High level view is that we want some spatial correlation in the group labels.
And also in the species counts. But the group and species counts will be totally
unrelated.
- The way we simulate this is to simulate two gaussian processes, as in the
mantel test paper. Since groups are categorical, we'll pass the GP through a
logistic function.
- We'll also consider making the $x$'s look like species counts. We do that
using a gaussian process under a poisson model.

\subsection{Methods description}
\label{subsec:methods_description}

adonis
- We will include $u$ as a covariate. This is how people would typically
recommend controlling for the spatial correlation, as far as I can tell.
- The basic idea of the test is to compute within and between distances in $x$,
where the groups are determined by $y$. The testing scheme permutes the labels
of $y$. Note that this breaks all correlation structure among the $y$s.

logistic regression
- We can use the species counts as predictors of $y$. We don't get a global
$p$-value for $y$, like in adonis, but we can get $p$-values for each of the
species breakdowns on their own.

linear model
- For a closer parallel with adonis, we can treat the category as a covariate.
Then, we can model the response label using a linear model. In the Poisson
generating scheme, we use a Poisson link.

\section{Results}
\label{sec:results}

give the summary figure
in this setup, adonis is not giving valid p-values
note that the logistic regression case is returning many more p-values: it has
one for each species in our simulation.

\bibliographystyle{plainnat}
\bibliography{refs.bib}

\end{document}
