\documentclass{article}
\usepackage{natbib}
\usepackage{graphicx}
\usepackage{amssymb, amsmath, amsfonts}
\usepackage[newfloat]{minted}
\usepackage{caption}
\newenvironment{code}{\captionsetup{type=listing}}{}
\SetupFloatingEnvironment{listing}{name=Source Code}
% ------------------------------------------------------------------------
% Packages
% ------------------------------------------------------------------------
\usepackage{amsmath}

% ------------------------------------------------------------------------
% Macros
% ------------------------------------------------------------------------
%~~~~~~~~~~~~~~~
% List shorthand
%~~~~~~~~~~~~~~~
\newcommand{\BIT}{\begin{itemize}}
\newcommand{\EIT}{\end{itemize}}
\newcommand{\BNUM}{\begin{enumerate}}
\newcommand{\ENUM}{\end{enumerate}}
%~~~~~~~~~~~~~~~
% Text with quads around it
%~~~~~~~~~~~~~~~
\newcommand{\qtext}[1]{\quad\text{#1}\quad}
%~~~~~~~~~~~~~~~
% Shorthand for math formatting
%~~~~~~~~~~~~~~~
\newcommand\mbb[1]{\mathbb{#1}}
\newcommand\mbf[1]{\mathbf{#1}}
\def\mc#1{\mathcal{#1}}
\def\mrm#1{\mathrm{#1}}
%~~~~~~~~~~~~~~~
% Common sets
%~~~~~~~~~~~~~~~
\def\reals{\mathbb{R}} % Real number symbol
\def\integers{\mathbb{Z}} % Integer symbol
\def\rationals{\mathbb{Q}} % Rational numbers
\def\naturals{\mathbb{N}} % Natural numbers
\def\complex{\mathbb{C}} % Complex numbers
\def\simplex{\mathcal{S}} % Simplex
%~~~~~~~~~~~~~~~
% Common functions
%~~~~~~~~~~~~~~~
\renewcommand{\exp}[1]{\operatorname{exp}\left(#1\right)} % Exponential
\def\indic#1{\mbb{I}\left({#1}\right)} % Indicator function
\providecommand{\argmax}{\mathop\mathrm{arg max}} % Defining math symbols
\providecommand{\argmin}{\mathop\mathrm{arg min}}
\providecommand{\arccos}{\mathop\mathrm{arccos}}
\providecommand{\asinh}{\mathop\mathrm{asinh}}
\providecommand{\dom}{\mathop\mathrm{dom}} % Domain
\providecommand{\range}{\mathop\mathrm{range}} % Range
\providecommand{\diag}{\mathop\mathrm{diag}}
\providecommand{\tr}{\mathop\mathrm{tr}}
\providecommand{\abs}{\mathop\mathrm{abs}}
\providecommand{\card}{\mathop\mathrm{card}}
\providecommand{\sign}{\mathop\mathrm{sign}}
\def\rank#1{\mathrm{rank}({#1})}
\def\supp#1{\mathrm{supp}({#1})}
%~~~~~~~~~~~~~~~
% Common probability symbols
%~~~~~~~~~~~~~~~
\def\E{\mathbb{E}} % Expectation symbol
\def\Earg#1{\E\left[{#1}\right]}
\def\Esubarg#1#2{\E_{#1}\left[{#2}\right]}
\def\P{\mathbb{P}} % Probability symbol
\def\Parg#1{\P\left({#1}\right)}
\def\Psubarg#1#2{\P_{#1}\left[{#2}\right]}
\def\Cov{\mrm{Cov}} % Covariance symbol
\def\Covarg#1{\Cov\left[{#1}\right]}
\def\Covsubarg#1#2{\Cov_{#1}\left[{#2}\right]}
\def\Var{\mrm{Var}}
\def\Vararg#1{\Var\left(#1\right)}
\def\Varsubarg#1#2{\Var_{#1}\left(#2\right)}
\newcommand{\family}{\mathcal{P}} % probability family
\newcommand{\eps}{\epsilon}
\def\absarg#1{\left|#1\right|}
\def\msarg#1{\left(#1\right)^{2}}
\def\logarg#1{\log\left(#1\right)}
%~~~~~~~~~~~~~~~
% Distributions
%~~~~~~~~~~~~~~~
\def\Gsn{\mathcal{N}}
\def\Ber{\textnormal{Ber}}
\def\Bin{\textnormal{Bin}}
\def\Unif{\textnormal{Unif}}
\def\Mult{\textnormal{Mult}}
\def\Cat{\textnormal{Cat}}
\def\Gam{\textnormal{Gam}}
\def\InvGam{\textnormal{InvGam}}
\def\NegMult{\textnormal{NegMult}}
\def\Dir{\textnormal{Dir}}
\def\Lap{\textnormal{Laplace}}
\def\Bet{\textnormal{Beta}}
\def\Poi{\textnormal{Poi}}
\def\HypGeo{\textnormal{HypGeo}}
\def\GEM{\textnormal{GEM}}
\def\BP{\textnormal{BP}}
\def\DP{\textnormal{DP}}
\def\BeP{\textnormal{BeP}}
%~~~~~~~~~~~~~~~
% Theorem-like environments
%~~~~~~~~~~~~~~~

%-----------------------
% Probability sets
%-----------------------
\newcommand{\X}{\mathcal{X}}
\newcommand{\Y}{\mathcal{Y}}
\newcommand{\D}{\mathcal{D}}
\newcommand{\Scal}{\mathcal{S}}
%-----------------------
% vector notation
%-----------------------
\newcommand{\bx}{\mathbf{x}}
\newcommand{\by}{\mathbf{y}}
\newcommand{\bt}{\mathbf{t}}
\newcommand{\xbar}{\overline{x}}
\newcommand{\Xbar}{\overline{X}}
\newcommand{\tolaw}{\xrightarrow{\mathcal{L}}}
\newcommand{\toprob}{\xrightarrow{\mathbb{P}}}
\newcommand{\laweq}{\overset{\mathcal{L}}{=}}
\newcommand{\F}{\mathcal{F}}

\linespread{1.25}

\title{Studying Species-Environment Associations among Correlated Sites}
\author{Kris Sankaran}

\begin{document}
\maketitle

On occasion, the $p$-values obtained by the permutation procedure described in
\citep{anderson2001new} (implemented as the \texttt{adonis} function in
\cite{oksanen2007vegan}) have seemed suspiciously significant, sometimes out of
line with other analysis on the same data. We have had various hypotheses about
why the permutation procedure might not be appropriate in microbiome data -- in
particular, bacteria and sites tend to be correlated across phyla and time. In
this note, we try to formalize this intuition and quantify the robustness (or
lack thereof) of the adonis permutation procedure. We believe that a more
careful understanding of this method is a relevant area of statistical research,
in light of the popularity of this method in microbiome analysis practice (see
e.g. \cite{fukuyama2012comparisons} and references therein).

Our analysis is directly inspired by \citep{guillot2013dismantling}, which
studied validity of the Mantel test, another type of distanced-based,
nonparametric hypothesis test. The main idea in both their work and ours is that
measurements across multiple sites will often include some sort of between-site
correlation structure, and that it is important to maintain this structure under
the permutation null. More precisely, the null hypothesis should only be that
the two processes are independent of one another, not that the two processes are
independent and each have no between-site correlation. But the latter structure
is what is simulated in adonis' permutation null, and if the processes are
actually independent but with within-process site correlation, adonis will
report that there is an association between them, when there are in fact none.

We will describe some simple linear-model based alternatives to adonis that that
seem much better behaved. These alternatives also cannot be applied when only
distances between sites is available -- a setting which is meaningful in
microbiome studies, where working entirely with UniFrac distances
\citep{lozupone2005unifrac} is common, for example.

There are three main limitations in this note,
\begin{itemize}
\item As in any simulation study, the conclusions are only as useful as the
  simulation mechanism is representative of the data encountered in practice.
  Ideally we could characterize the plausibility of our simulation mechanism,
  with respect to real data.
\item None of our alternatives are quite satisfactory, because either (1) they
  assume little knowledge about the data generating mechanism, but provide
  invalid $p$-values, or (2) they provide valid $p$-values, but require
  substantial knowledge of the data generating mechanism.
\item We do not provide well-behaved alternatives to adonis when a noneuclidean
  distance across species is desired.
\end{itemize}

Hopefully these deficiencies will be resolved as this work matures.

\section{Setup}
\label{sec:setup}

In this section, we describe the our simulation design. This has two components:
how do we generate the null data, and how do we test for associations. We have
several approaches for each.

\subsection{Data generating mechanism}
\label{subsec:data_generating_mechanism}

As mentioned in the introduction, the goal of our simulation study is to
formally encode the qualitative observation that adonis may incorrectly report
an association between independent measurements when the measurements exhibit
between-site correlation. Suppose $X \in \reals^{n \times p}$ is a matrix of
abundances for $p$ species across $n$ sites. Let $y \in \{0, 1\}^{n}$ encode a
binary categorical measurement for each site. For example, this could represent
whether a site is arid or not, and the study might be attempting to determine
whether the species signatures across sites is related to site climate.

In our first data generating mechanism, we imagine that each species has an
abundance that varies smoothly over space. Specifically, we model the abundance
of the $j^{th}$ species over space according to a Gaussian Process (GP) with
kernel
$\kappa\left(u_{i}, u_{j}\right) = a\exp{-\frac{1}{2\sigma^{2}}\|u_{i} - u_{j}\|^{2}}$.
We think of $u \in \reals^{2}$ as the geographic location of the site.
Specifically, we suppose,
\begin{align}
  x_{j} &\sim GP\left(0, \kappa\right)
\end{align}
independently for $j = 1, \dots, p$. This defines $p$ random functions, one for
each species, which have high values at locations $u$ where the the species are
abundant and low values where they are not. See Figure \ref{fig:x_gaussian} for
a representative of the simulated $X$.

\begin{figure}
  \centering
  \includegraphics[width=0.9\textwidth]{figure/x_gaussian}
  \caption{\label{fig:x_gaussian} The simulated $X$ matrix according to the
    first data generating mechanism. Each panel represents one species. Within a
    panel, a point gives the abundance for the associated species at one site
    and at one timepoint. The $x$-axis encodes the coordinate $u_{1}$ of the
    sites location, while the size of the points encodes $u_{2}$. The color of
    each point encodes the site characteristic $y$. The main idea is that
    species abundances vary smoothly across space and are independent of site
    characteristics.}
\end{figure}

For the site labels $y$, we first define a random process of probabilities by
passing an independent GP through a logit link,
\begin{align}
  p \vert f &\sim \frac{1}{1 + \exp{-f}} \\
  f &\sim GP\left(0, \kappa\right).
\end{align}

\begin{figure}[ht]
  \centering
  \includegraphics[width=0.9\textwidth]{figure/p_gp}
  \caption{\label{fig:p_gp} To simulate the vector of site characteristics $y$,
    we pass a GP through a logit link. Each circle represents the probability
    that a given site will have $y = 1$. The $x$-axis and sizes encode $u_{1}$
    and $u_{2}$, respectively, as in Figure \ref{fig:x_gaussian}. The observed
    characteristics are marked as crosses. The idea is that the probability of
    taking on $y = 1$ varies somewhat smoothly over space.}
\end{figure}

We suppose our data set are drawn at $n$ locations $u_{1}, \dots, u_{n}$, and
that $X$ has $ij^{th}$ entry $x_{j}\left(u_{i}\right)$ and the $i^{th}$ element
of $y$ is drawn as $y_{i} \sim \Ber\left(p\left(u_{i}\right)\right)$. See Figure
\ref{fig:p_gp} for a visual representative of this process. If we use different
realizations of a GP for generating the $x_{j}$ and $f$, then the site category
labels $y$ and species abundances $X$ are independent of one another. However,
across sites, both $y$ and the $x_{j}$'s have strong between-site correlation.
Therefore, we have provided one quantitative model of the null situation
described in the introduction.

\begin{figure}
  \centering
  \includegraphics[width=0.9\textwidth]{figure/x_poisson}
  \caption{\label{fig:x_poisson} The second data generation mechanism ensures
    that $X$ are true counts. The figure is read exactly as in figure
    \ref{fig:x_gaussian}. Again, we see the average count varies smoothly over
    space, though there tend to be larger swaths of space where all counts are
    small.}
\end{figure}

Our second data generating mechanism is a minor variant of the first, where
we ensure that the entries of $X$ are counts. To achieve this, we pass a GP
through a Poisson link. That is, the mechanism is identical to the above,
except that we take,
\begin{align}
  g &\sim GP\left(0, \kappa\right)
\end{align}
and at location $u_{i}$ we draw
$x_{j}\left(u_{i}\right) \sim \Poi\left(\exp{g\left(u_{i}\right)}\right)$.

\subsection{Methods description}
\label{subsec:methods_description}

We review three approaches to measuring the strength of the association between
$X$ and $y$ -- adonis, logistic regression, and linear modeling -- which we
later apply in our simulation study.

\subsubsection{Adonis}
\label{subsubsec:adonis}

Adonis is a nonparametric alternative to ANOVA \citep{anderson2001new}. The main
idea is that the within and between sums-of-squares, which are central to ANOVA,
can be represented in terms of pairwise sums-of-squares. Substituting generic
pairwise distances for these pairwise sums-of-squares leads to a nonparametric
analog of the ANOVA decomposition. The advantage of this approach is that it
allows inference based simply on pairwise distances between sites -- say,
Bray-Curtis or UniFrac distances -- which might be more sensitive to the types
of variation that are meaningful in the microbial ecology setting than ordinary
Euclidean distance, which is implicitly assumed by ANOVA.

However, the nonparametric analog of the $F$-statistic is no longer
$F$-distributed, so standard theory does not provide valid inference. Instead, a
permutation scheme is considered. In the one-way ANOVA case, we recompute the
nonparametric $F$-statistic under many permutations of $y$ and use the resulting
histogram as the reference null distribution.

When more covariates are included, such a permutation scheme is not sufficient,
and in general it is not straightforwards to design valid permutation tests,
because the permutations must respect constraints imposed by the null hypothesis
\cite{anderson2001new}. The implementation in \citep{oksanen2007vegan} is only
approximate, applying independent permutations to each covariate in the model.
Note in particular that this permutation scheme breaks the between-site
correlation in the simulated $y$.

%% \begin{align}
%%   WSS &= \frac{1}{n} \sum_{k = 1}^{K} \sum_{i \in S_{k}} \left(x_{i} - \bar{x}_{k}\right)^{2}
%% \end{align}

%% \begin{align}
%%   BSS &= \frac{1}{K} \sum_{k = 1}^{K} \left(\bar{x}_{k} - \bar{x}\right)^{2}
%% \end{align}


We apply adonis using the function \texttt{adonis} in the vegan package
\citep{oksanen2007vegan}. In our simulation, we treat the distances between
species profiles $X$ as the response and the category $y$ and locations $u$ as
covariates. The associations with categories $y$ is of core interest, while the
$u$s are included to ``control for'' location variation\footnote{We will see
  that this does not seem to control for the (nonlinear) location variation seen
  in our simulation.}. Specifically, our calls have the form given in code block
\ref{code:adonis}, where we use a euclidean distance for the ordinary GP setup
and Bray-Curtis for the Poisson setup.

\begin{code}
\begin{minted}[mathescape, linenos, frame=lines, framesep=2mm]{R}
  ## use method = 'euclidean' for gaussian process x
  adonis(x ~ y + u, method = "bray")
\end{minted}
\captionof{listing}{An example call to adonis in our simulation. Here,
  \texttt{x} is an $n \times p$ matrix of species abundances across $n$ sites,
  $y$ is a length $n$ binary indicator variable, and $u$ is an $\reals^{n \times
    2}$ matrix of site coordinates.
}
\label{code:adonis}
\end{code}

\subsection{Logistic regression}
\label{subsec:logistic_regression}

We could instead measure the association between species counts $X$ and site
characteristics $y$ using a logistic regression. The fact that the $y$ are
binary means we can treat them as a response in a logistic regression model,
using $X$ and site locations $u$ as covariates (again, $u$ is thrown in as a
control in the association between $x$ and $y$). Note that this approach does
not apply any specialized distances between species profiles, which is one
reason why adonis might be preferred.

We can recover $p$-values using standard theory for logistic regression. This
theory does not quite apply to our simulation setting, however, because the
effect of the $u$ enters nonlinearly, resulting in correlated errors. In our
simulation, different coordinates of $y_{i}$ have nonlinearly varying
probabilities depending on their location.

We implement this using the \texttt{glm} command, as in code block
\ref{code:logistic}.

\begin{code}
\begin{minted}[mathescape, linenos, frame=lines, framesep=2mm]{R}
  glm(y ~ x + u, family = binomial())
\end{minted}
\captionof{listing}{An example call for applying logistic regression of species
  counts onto site characteristic.}
\label{code:logistic}
\end{code}

\subsection{Linear regression}
\label{subsec:linear_regression}

Alternatively, we could flip the species counts and site characteristics in our
regression. Unlike adonis, we can only have one species at a time in our
response variable. In our simulation, we will only consider a model applied to
the first species. In the GP data-generating mechanism, we will use a standard
linear regression, while for the Poisson-GP data-generating mechanism, we will
use a Poisson regression.

We use the $p$-value for the coefficient of $y$ provided by the standard
regression theory. Note that, as in the logistic regression, the errors
$\eps_{i}$ are actually correlated in our simulation, so the theory is in fact
violated. We will see what effect this has on the validity of our test during
our simulation study. Code for this procedure is provided in code block
\ref{code:linear_reg}.

\begin{code}
\begin{minted}[mathescape, linenos, frame=lines, framesep=2mm]{R}
  glm(x[, 1] ~ y + u, family = "poisson")
\end{minted}
\captionof{listing}{An example call for applying Poisson regression of site
  characteristics onto counts for the first species.}
\label{code:linear_reg}
\end{code}

\subsection{Generalized Least Squares}
\label{subsec:generalized_least_squares}

Ideally, we would be able to produce valid $p$-values without assuming too much
knowledge of the underlying data generating mechanism. To this end, we consider
applying a generalized least squares (GLS) to the regression of the first
species abundances onto the site characteristic. The assumed model has the form
$x_{i1} = \beta_{y_{i}} + \eps_{i}$ where
$\left(\eps_{1}, \dots, \eps_{n}\right)^{T} \sim \Gsn\left(0, \Sigma\right)$.
This addresses the correlation between errors, but not the nonzero means
resulting from model misspecification. From the simulation results, we can see
that this remaining misspecification is enough to keep the $p$-values from being
uniform.

\subsection{Generalized Additive Models}
\label{subsec:generalized_additive_models}

To this point, we have avoided using knowledge about the smoothly varying nature
of the species abundances and site characteristic probabilities. We had been
hoping to derive valid procedures without too much problem-specific knowledge.
While we have succeeded in finding procedures that are much more robust than
adonis, none of those we have described so far have been completely valid. On
the other hand, if we had identified this particular problem structure, we we
would be able to build it into our modeling strategy. This results in both a
better fitting model and valid hypothesis tests.

To this end, we will consider a generalized additive model (GAM) with nonlinear
interactions terms for the effect of sites on species abundances. This exactly
reflects the true smoothly varying nature of abundances across sites, built into
the GP data generating mechanism. The code for this procedure is provided in
code block \ref{code:gam}.

If the estimated GAM influence of site location $u$ on species abundances $x$
match the true underlying GP, then the errors will indeed by i.i.d. normals, so
the assumptions required for testing will be fulfilled.

Whether this much structural knowledge will ever be available in practical data
analysis is unclear, and the possibility that it might not available or might
not be correctly specified is the main deficiency of this approach.

\begin{code}
\begin{minted}[mathescape, linenos, frame=lines, framesep=2mm]{R}
gam(
  x[, 1] ~ y + ti(u[, 1]) + ti(u[, 2]) + ti(u[, 1], u[, 2]),
  family = "poisson"
)
\end{minted}
\captionof{listing}{An example call for Poisson regression with a smooth
  influence of the site location on abundance. The first two \texttt{ti} terms
  correspond to nonlinear main effects of site location, while the third
  provides an interaction. The corresponding call for gaussian response uses a
  \texttt{gaussian} family argument.}
\label{code:gam}
\end{code}

\subsection{MDS follows by GAM}
\label{subsec:mds_followed_by_gam}

The primary appeal of adonis in the microbiome literature is that it allows the
use of a distance matrix computed across many species as a ``response''
variable. Generally, there is an interest in whether certain site
characteristics might induce different species profiles, depending on their
value, and using a species-profile based distance matrix serves as a type of
proxy for this.

The only approach we have found so far that we would feel comfortable
recommending to practitioners is the GAM-based approach\footnote{That was used a
  GAM here is not particularly important. Any method that captured smooth
  geographic variation across sites in our simulation would have suffied.}:
large-scale (e.g., geographic) site variation in species abundances should be
modeled before quantifying the influence of other site characteristics.

The main barrier that keeps us from achieving these two goals simultaneously is
that there are few (if any?) flexible modeling approaches that use distances
matrices as a ``response variable''. Most modeling techniques work with a vector
of responses. However, this also suggests one basic idea for blending
perspectives without designing entirely new algorithms -- we can embded
arbitrary distances into some euclidean space, and then apply modeling
techniques in this artificial space.

Concretely, we consider the following meta-algorithm,

\begin{itemize}
\item Compute a distance matrix between sites' species abundance profiles, using
  any distance of interest.
\item Embed this distance matrix into a $K$-dimensional euclidean space, using
  multidimensional scaling. 
\item Perform either a single $K$-dimensional multiresponse regression or $K$
  separate single-response regressions, paying attention to the underlying site
  variation, as in the GAM example above.
\end{itemize}

The optimal choices of distance, $K$, and final regression strategy are not
entirely clear, and the choice is left to the modeler. Note that a separate
coefficient is estimated for each response dimension -- this can complicate
interpretation of the influence of individual factors on site profiles. However,
displaying the intermediate MDS plot can help -- it might turn out that
different groups of species vary along different MDS dimensions, and a site
characteristic might only be associated with the abundances of one of these
groups of species.

Since the GAM approach was successful in the last example, we use it again
within this MDS approach. The implementation we apply is provided in code block
\ref{code:mds_gam}, where the distance is either ``bray'' or ``euclidean''
depending on the Poisson or Gaussian setup.

\begin{code}
\begin{minted}[mathescape, linenos, frame=lines, framesep=2mm]{R}
mds_lm <- function(x, y, u, K = 3, ...) {
  mds_x <- cmdscale(vegdist(x, ...), k = K)
  models <- list()
  for (k in seq_len(K)) {
    models[[k]] <- gam(
      mds_x[, k] ~ y + ti(u[, 1]) + ti(u[, 2]) + ti(u[, 1], u[, 2])
    )
  }
  models
}
\end{minted}
\captionof{listing}{The function we use to do inference on site characteristic
  across entire species profiles, using the MDS + GAM approach. The idea is that
  the $K$ MDS embedded dimensions are each used as responses for $K$ separate
  GAM fits, where the underlying MDS scores are assumed to vary smoothly as a
  function of site location.}
\label{code:gam}
\end{code}

\section{Results}
\label{sec:results}

For each data generating mechanism and method described in section
\ref{sec:setup}, we simulate 1000 replicates. The resulting $p$-values are
described in figure \ref{fig:pvals_comparison}.

More precisely, the adonis $p$-value is the permutation $p$-value for the $y$
covariate, and the (Poisson) linear model $p$-values those for the coefficient
of the $y$ covariate. In contrast, the $p$-values for the logistic regression
are those associated with coefficients for each species -- this is why there are
so many more $p$-values in the logistic regression row of figure
\ref{fig:pvals_comparison}.

\begin{figure}
  \centering
  \includegraphics[width=0.8\textwidth]{figure/pvals_comparison}
  \caption{\label{fig:pvals_comparison} A comparison of the null $p$-values
    according to the different data-generating mechanisms and methods. Different
    methods are arranged across rows, while the GP and Poisson-GP data
    generating mechanisms are in the left and right columns, respectively.
    Ideally, these would all look approximately uniform, and the fact that they
    are not means these tests are not valid.}
\end{figure}

Evidently, the $p$-values associated with all procedures are not quite uniform.
In light of our earlier discussion, this is not surprising. Both the logistic
and linear regression models fail because the errors are neither independent nor
identically distributed, since the location $u$ has a nonlinear influence on
both $x$ and $y$. The adonis permutation procedure also fails to account for the
influence of $u$, since it breaks the correlation between sites. However, those
provided by adonis are strongly concentrated on very low values. Indeed, the
permutation procedure in this setting seems poorly suited in the case of
correlated sites, even when they are included as covariates in the regression.

So, the essential takeaways of this study are,

\begin{itemize}
\item 
\end{itemize}

\bibliographystyle{plainnat}
\bibliography{refs.bib}

\end{document}
