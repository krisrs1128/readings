\documentclass{article}
\usepackage{natbib}
\usepackage{graphicx}
\input{preamble.tex}

\title{Sensitivity of ADONIS to between-site correlations}
\author{Kris Sankaran}
\maketitle

\begin{document}

- Purpose: This is directly inspired by dismantling mantel Idea: Permutation
design in adonis might be sensitive to between-site correlation structure, and
this might make us skeptical of what appear to be very significant results in
actual data analysis
- The permutation mechanism fails to maintain the between site correlation, and
this restriction really should be required by the null
- Under the null, the two processes are totally independent, but they have some
smooth spatially varying structure. In this setting, adonis will report an
association between the processes, even if there is none.
- Simple linear and logistic regression approaches to this problem are more or
less okay. They might not be totally valid, because of correlation between
samples, but it's much more robust than adonis.
- Limitation: The model is a toy model, only as useful as the simulation is
representative. Also, we're not proposing any more sophisticated approach to
this problem (e.g., knockoffs)

\section{Setup}
\label{sec:setup}

* data generation *

- What models do we consider simulating under?
- High level view is that we want some spatial correlation in the group labels.
And also in the species counts. But the group and species counts will be totally
unrelated.
- The way we simulate this is to simulate two gaussian processes, as in the
mantel test paper. Since groups are categorical, we'll pass the GP through a
logistic function.
- We'll also consider making the $x$'s look like species counts. We do that
using a gaussian process under a poisson model.

* methods review *

adonis
- We will include $u$ as a covariate. This is how people would typically
recommend controlling for the spatial correlation, as far as I can tell.
- The basic idea of the test is to compute within and between distances in $x$,
where the groups are determined by $y$. The testing scheme permutes the labels
of $y$. Note that this breaks all correlation structure among the $y$s.

logistic regression
- We can use the species counts as predictors of $y$. We don't get a global
$p$-value for $y$, like in adonis, but we can get $p$-values for each of the
species breakdowns on their own.

linear model
- For a closer parallel with adonis, we can treat the category as a covariate.
Then, we can model the response label using a linear model. In the Poisson
generating scheme, we use a Poisson link.

\section{Results}
\label{sec:results}

give the summary figure
in this setup, adonis is not giving valid p-values
note that the logistic regression case is returning many more p-values: it has
one for each species in our simulation.

\end{document}
