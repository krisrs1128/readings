\documentclass{article}
\usepackage{natbib}
\usepackage{graphicx}
\input{preamble.tex}

\title{Experiments with Nonparametric ANOVA}
\author{Kris Sankaran}

\begin{document}
\maketitle

Can we try to understand the method described in \citep{anderson2001new} from a
careful statistical point of view? In what settings is it appropriate? Is it
possible to characterize \textit{why} hypotheses get rejected, and provide a
sense of effect sizes, rather than focusing only on overall significance? Though
the method is based on ANOVA -- one of the most well-understood techniques in
statistics -- there remains something of an air of mystery surrounding ADONIS.
The need for a more careful study is compounded by the fact that, though the
original method was developed on a data that has a very different form from that
of microbiome studies, it has become particularly popular in the microbiome
analysis community. If for nothing else, it would be comforting to know that
ADONIS behaves similarly in both microbiome and classical ecological settings.
The goal of this note is to brainstorm and summarize some simulation and
mathetical experiments towards understanding the behavior of this method.

\section{Summary}

establish notation and describe the method

\section{Simulation ideas}

\subsection{Under the null}

By definition, under the null, the $p$-values created by ADONIS are uniform.
Nonetheless, it is still worth studying the permutation distribution of the
distance-based $F$-statistic under different data-generation scenarios. For
example, it would be useful to know whether there are any types of concentration
phenomena that arise under different $\frac{p}{n}$ regimes or how highly
non-normal, zero-inflated, or outlier-contaminated data affect the null.

We can approach these questions through comprehensive simulations. Here are some
parameters that might be worth considering.

\begin{itemize}
  \item Distributions: Consider gaussian, gamma, and negative binomial
    distributed response matrices $Y$. For each of these, we can consider
    variation across distributional parameters (does the method behave the same
    when counts are large vs. small?)
  \item Distances: What happens when we use euclidean, bray-curtis, unifrac distances, ...
  \item Dimensionality in $Y$: What happens when the ratio for the number of
    samples vs. species changes? What happens when, fixing this ratio, the total
    number of species and samples change?
  \item Dimensionality in $X$: What happens when we change the number of factors
    upon which we are looking for an association?
  \item Contamination: What happens when we contaminate with outliers, or add
    zero-inflation?
  \item Dependency: What happens when many of the species in $Y$ are correlated?
    For example, what if $Y$ is low-rank? What happens when the factors under
    consideration are correlated?
\end{itemize}

Under the ordinary gaussian case with euclidean distance, how does the
permutation $F$-statistic compare with that from standard MANOVA theory?

\subsection{Under alternatives}

Every data generation mechanism under the null model (count matrices $Y$
unrelated to factors $X$) can be studied in the case where there is an
association between the factors and the response. But it is also possible to
introduce further complexity, by varying the relationship between $X$ and $Y$.
For example, we can control
\begin{itemize}
\item Signal strength: Assuming a particular association between $X$ and $Y$, we
  can modify the underlying effect size.
\item Noise structure: What happens when we have low vs. high noise in the
  relationship between $X$ and $Y$?
\item Structural form: What happens when we assume a linear association? A
  nonlinear association? An association with some columns of $Y$, but not with
  others? An association with only a subset of samples?
\end{itemize}

\bibliographystyle{plainnat}
\bibliography{refs.bib}

\end{document}
